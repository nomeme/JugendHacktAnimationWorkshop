\subsection{Unser Projekt}

\begin{frame}
    \frametitle{Unser Projekt}
    \begin{itemize}
        \item Wir schauen uns heute einzelne Pixel in form von WS2812b LEDs an.
        \item Diese sind auch als \href{https://en.wikipedia.org/wiki/Adafruit_Industries\#NeoPixel}{NeoPixel} bekannt.
    \end{itemize}
\end{frame}

\begin{frame}
    \frametitle{Unser Projekt}
    \framesubtitle{NeoPixel}
    \begin{minipage}{0.5\textwidth}
        \begin{itemize}
            \item \href{https://en.wikipedia.org/wiki/Adafruit_Industries\#NeoPixel}{NeoPixel} besitzen wie ein echter Pixel drei Farben.
            \item Hier sind es LEDs mit den Farben Rot, Grün und Blau.
        \end{itemize}
    \end{minipage} \hfill
    \begin{minipage}{0.45\textwidth}
        \grabto{https://upload.wikimedia.org/wikipedia/commons/thumb/b/b6/Mini_NeoPixel.jpg/1920px-Mini_NeoPixel.jpg}{neopixel.jpg}
        \begin{figure}
            \includegraphics[width=\textwidth]{neopixel.jpg}
        \end{figure}
    \end{minipage}
\end{frame}

\begin{frame}
    \frametitle{Unser Projekt}
    \framesubtitle{NeoPixel}
    \begin{minipage}{0.5\textwidth}
        \begin{itemize}
            \item In den einzelnen \href{https://en.wikipedia.org/wiki/Adafruit_Industries\#NeoPixel}{NeoPixel} befindet sich ein kleiner Mikrochip.
            \item Der ermöglicht uns die einzelnen LEDs auf dem Pixel zu steuern und so die Farbe einzustellen.
        \end{itemize}
    \end{minipage} \hfill
    \begin{minipage}{0.45\textwidth}
        \grabto{https://upload.wikimedia.org/wikipedia/commons/thumb/b/b6/Mini_NeoPixel.jpg/1920px-Mini_NeoPixel.jpg}{neopixel.jpg}
        \begin{figure}
            \includegraphics[width=\textwidth]{neopixel.jpg}
        \end{figure}
    \end{minipage}
\end{frame}

\begin{frame}
    \frametitle{Unser Projekt}
    \framesubtitle{NeoPixel}
    \begin{minipage}{0.5\textwidth}
        \begin{itemize}
            \item Sie werden oft als \href{https://en.wikipedia.org/wiki/LED_strip_light}{LED Strip} angeboten bei dem sich jede LED einzeln einfärben lässt.
        \end{itemize}
    \end{minipage} \hfill
    \begin{minipage}{0.45\textwidth}
        \grabto{https://upload.wikimedia.org/wikipedia/commons/thumb/b/b6/Mini_NeoPixel.jpg/1920px-Mini_NeoPixel.jpg}{neopixel.jpg}
        \begin{figure}
            \includegraphics[width=\textwidth]{neopixel.jpg}
        \end{figure}
    \end{minipage}
\end{frame}

\begin{frame}
    \frametitle{Unser Projekt}
    \framesubtitle{Aufbau}
    \begin{minipage}{0.5\textwidth}
        \begin{itemize}
            \item Jeder \href{https://en.wikipedia.org/wiki/Adafruit_Industries\#NeoPixel}{NeoPixel} hat vier Anschlüsse.
            \item Zwei Anschlüsse für die Stromversorgung (+ und -)
            \item Einen Eingang für die Steuersignale
            \item Einen Ausgang für die Steuersignale
        \end{itemize}
    \end{minipage} \hfill
    \begin{minipage}{0.45\textwidth}
        \grabto{https://upload.wikimedia.org/wikipedia/commons/thumb/b/b6/Mini_NeoPixel.jpg/1920px-Mini_NeoPixel.jpg}{neopixel.jpg}
        \begin{figure}
            \includegraphics[width=\textwidth]{neopixel.jpg}
        \end{figure}
    \end{minipage}
\end{frame}

\begin{frame}
    \frametitle{Unser Projekt}
    \framesubtitle{Aufbau}
    \begin{minipage}{0.5\textwidth}
        \begin{itemize}
            \item Die Pixel leiten die Steuersignale immer an den nächsten Pixel weiter.
            \item So können wir viele \href{https://en.wikipedia.org/wiki/Adafruit_Industries\#NeoPixel}{NeoPixel} in einer Kette hintereinander anschließen.
            \item Wir können diese so einfärben und Animationen erzeugen.
        \end{itemize}
    \end{minipage} \hfill
    \begin{minipage}{0.45\textwidth}
        \grabto{https://upload.wikimedia.org/wikipedia/commons/thumb/b/b6/Mini_NeoPixel.jpg/1920px-Mini_NeoPixel.jpg}{neopixel.jpg}
        \begin{figure}
            \includegraphics[width=\textwidth]{neopixel.jpg}
        \end{figure}
    \end{minipage}
\end{frame}

