\subsection{Moderne Anzeigen}

\begin{frame}
    \frametitle{Moderne Anzeigen}
    \begin{itemize}
        \item Hier schauen wir uns Anzeigemöglichkeiten an, die auf sogenannten \href{https://de.wikipedia.org/wiki/Pixel}{Pixeln} basieren.
    \end{itemize}
\end{frame}

\begin{frame}
    \frametitle{Moderne Anzeigen}
    \framesubtitle{Pixel}
    \begin{minipage}{0.5\textwidth}
        \begin{itemize}
            \item Ein \href{https://de.wikipedia.org/wiki/Pixel}{Pixel} ist ein Punkt, der einen Farbwert annehmen kann.
            \item Der Farbwert entsteht durch die Kombination der Farbwerte Rot, Grün und Blau.
        \end{itemize}
    \end{minipage} \hfill
    \begin{minipage}{0.45\textwidth}
        \grabto{https://upload.wikimedia.org/wikipedia/commons/6/65/TVPixel.jpg}{pixel.jpg}
        \begin{figure}
            \includegraphics[width=\textwidth]{pixel.jpg}
        \end{figure}
    \end{minipage}
\end{frame}

\begin{frame}
    \frametitle{Moderne Anzeigen}
    \framesubtitle{Pixel}
    \begin{minipage}{0.5\textwidth}
        \begin{itemize}
            \item Heutige Monitore und Fernseher stellen Bilder mit Hilfe von Millionen von einzelnen Pixeln dar.
            \item Wie die einzelnen Farbwerte im Pixel entstehen hängt dabei von der jeweiligen Technologie ab.
        \end{itemize}
    \end{minipage} \hfill
    \begin{minipage}{0.45\textwidth}
        \grabto{https://upload.wikimedia.org/wikipedia/commons/6/65/TVPixel.jpg}{pixel.jpg}
        \begin{figure}
            \includegraphics[width=\textwidth]{pixel.jpg}
        \end{figure}
    \end{minipage}
\end{frame}

\begin{frame}
    \frametitle{Moderne Anzeigen}
    \framesubtitle{CRT Monitor}
    \begin{minipage}{0.5\textwidth}
        \begin{itemize}
            \item Beim \href{https://de.wikipedia.org/wiki/Kathodenstrahlr\%C3\%B6hrenbildschirm}{CRT Monitor} tastet ein über Elektromagnete gesteuerter Elektronenstrahl einzelne Leuchtpunkte auf dem Leuchtschirm der \href{https://de.wikipedia.org/wiki/Kathodenstrahlr\%C3\%B6hre}{Kathodenstrahlröhre} ab.
        \end{itemize}
    \end{minipage} \hfill
    \begin{minipage}{0.45\textwidth}
        \grabto{https://upload.wikimedia.org/wikipedia/commons/thumb/6/66/Cathode_ray_tube_de.svg/1920px-Cathode_ray_tube_de.svg.png}{crt.png}
        \begin{figure}
            \includegraphics[width=\textwidth]{crt.png}
        \end{figure}
    \end{minipage}
\end{frame}

\begin{frame}
    \frametitle{Moderne Anzeigen}
    \framesubtitle{CRT Monitor}
    \begin{minipage}{0.5\textwidth}
        \begin{itemize}
            \item Dabei fängt er oben links an und tastet sich dann Zeile für Zeile nach unten hin weiter.
            \item Wenn der Elektronenstrahl auf die Leuchtschicht trifft, regt er sie zum Leuchten an und so entsteht nach und nach ein Bild.
        \end{itemize}
    \end{minipage} \hfill
    \begin{minipage}{0.45\textwidth}
        \grabto{https://upload.wikimedia.org/wikipedia/commons/thumb/6/66/Cathode_ray_tube_de.svg/1920px-Cathode_ray_tube_de.svg.png}{crt.png}
        \begin{figure}
            \includegraphics[width=\textwidth]{crt.png}
        \end{figure}
    \end{minipage}
\end{frame}

\begin{frame}
    \frametitle{Moderne Anzeigen}
    \framesubtitle{LCD Monitor}
    \begin{minipage}{0.5\textwidth}
        \begin{itemize}
            \item Bei einem \href{https://de.wikipedia.org/wiki/Fl\%C3\%BCssigkristallanzeige}{LCD Bildschirm} kommt eine flächige Hintergrundbeleuchtung zum Einsatz.
            \item In jedem Pixel gibt es dann einen Filter für die Farben Rot, Grün und Blau.
        \end{itemize}
    \end{minipage} \hfill
    \begin{minipage}{0.45\textwidth}
        \grabto{https://upload.wikimedia.org/wikipedia/commons/c/c0/Lcd.jpg}{lcd.jpg}
        \begin{figure}
            \includegraphics[width=\textwidth]{lcd.jpg}
        \end{figure}
    \end{minipage}
\end{frame}

\begin{frame}
    \frametitle{Moderne Anzeigen}
    \framesubtitle{LCD Monitor}
    \begin{minipage}{0.5\textwidth}
        \begin{itemize}
            \item Je mehr spannung an dem Filter anliegt, desto weniger Licht lässt er durch.
            \item So entsteht am Ende ein ganz bestimmter Farbwert entsteht.
            \item Fun fact: Schwarz braucht daher bei einem LCD Bildschirm am meisten Energie, da alle Filter voll angeschaltet sein müssen.
        \end{itemize}
    \end{minipage} \hfill
    \begin{minipage}{0.45\textwidth}
        \grabto{https://upload.wikimedia.org/wikipedia/commons/c/c0/Lcd.jpg}{lcd.jpg}
        \begin{figure}
            \includegraphics[width=\textwidth]{lcd.jpg}
        \end{figure}
    \end{minipage}
\end{frame}

\begin{frame}
    \frametitle{Moderne Anzeigen}
    \framesubtitle{OLED}
    \begin{minipage}{0.5\textwidth}
        \begin{itemize}
            \item Bei einem \href{https://de.wikipedia.org/wiki/Organische_Leuchtdiode}{OLED Bildschirm} werden organische Leuchtdioden verwendet.
            \item Im Gegensatz zu \href{https://de.wikipedia.org/wiki/Fl\%C3\%BCssigkristallanzeige}{LCD Bildschirmen} erzeugen hier kleine organische LEDs das Licht.
        \end{itemize}
    \end{minipage} \hfill
    \begin{minipage}{0.45\textwidth}
        \grabto{https://upload.wikimedia.org/wikipedia/commons/8/89/Nexus_one_screen_microscope.jpg}{oled.jpg}
        \begin{figure}
            \includegraphics[width=\textwidth]{oled.jpg}
        \end{figure}
    \end{minipage}
\end{frame}

\begin{frame}
    \frametitle{Moderne Anzeigen}
    \framesubtitle{OLED}
    \begin{minipage}{0.5\textwidth}
        \begin{itemize}
            \item Vorteil: Schwarz ist tatsächlich Schwarz und benötigt keinen Strom, da die kleinen OLEDs einfach ausgeschaltet werden.
            \item Nachteil: deutlich leuchtschwächer als herkömmliche LEDs und eine wesentlich kürzere Lebensdauer.
        \end{itemize}
    \end{minipage} \hfill
    \begin{minipage}{0.45\textwidth}
        \grabto{https://upload.wikimedia.org/wikipedia/commons/8/89/Nexus_one_screen_microscope.jpg}{oled.jpg}
        \begin{figure}
            \includegraphics[width=\textwidth]{oled.jpg}
        \end{figure}
    \end{minipage}
\end{frame}

